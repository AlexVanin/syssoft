\begin{defi}{Прерывание}
	это приостановка выполнения программы по специальному сигналу.
\end{defi}

Механизм прерываний нужен для оповещения системы об асинхронных событиях, таких, как нажатие клавиши на клавиатуре.

При выработке прерываний выполняется переход в режим ядра и поиск обработчика прерывания в IDT --- Interrupt Descriptor Table. После исполнения обработчика прерываний выполняется обратный переход --- в режим задачи.

Разница между исключениями и прерываниями состоит в том, что исключения вырабатываются процессором, а прерывания --- периферией.
