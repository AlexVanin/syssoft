Как до популяризации графических оболочек, так и после, базовой пользовательской средой является терминал.

В общем случае, терминал --- это точка входа пользователя в систему, обладающая способностью передавать текстовую информацию. Операционная система обменивается с терминалом через последовательный интерфейс - терминальную линию.

В роли терминала может выступать как отдельное устройство, так и программа. Нам важно знать то, что при включении терминала запускается процесс авторизации пользователя, после которого запускается командный интерпретатор.

Терминал принимает текст и управляющие последовательности от устройств ввода (например, клавиатуры) и передает их командному интерпретатору.