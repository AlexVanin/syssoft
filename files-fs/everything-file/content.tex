В современном мире существует огромное количество самых различных устройств, предназначенных для ввода и вывода информации. Было бы тяжело, если бы с каждым из этих устройств пришлось работать по-разному. Для решения этой проблемы создателями UNIX была выбрана концепция “Everything is a file“ или “Всё есть файл“.

В этой концепции, всё, с чем бы не взаимодействовала пользовательская программа, представляется для этой программе в виде файла. Это даёт следующие возможности: мы просто пользуемся принтером, жестким диском или другим устройством как обычным файлом --- пишем в него данные с помощью одного и того же файлового интерфейса, одних и тех же системных вызовов. В свою очередь, операционная система направляет данные по назначению.

Долгие споры приводят к тому, что четкого определения слову файл нет. В нашем курсе будем полагать, что:

\begin{defi}{Файл}
	совокупность неких данных и метаинформации (метаданных), которая описывает эти данные.
\end{defi}

\textit{Не будем претендовать, что оно всегда и везде будет соответствовать действительности, потому что на курсе операционных систем вам скажут, что файлом может называться, например, магнитный кусок ленты, на которой вы записали данные. И будут, безусловно, правы. Поэтому для себя запомним, что файл это некая такая абстрактная неведомая вещь, с которой мы можем условно каким-то образом взаимодействовать и по сути она нужна для работы с данными.}

\begin{important}
	В заголовке есть некоторое уточнение - что всё есть файл, кроме потоков и ядра. Дело в том, что в юниксе еще есть много других вещей в том числе потоки (имеются в виду не streams, а threads) и ядро. Ядро само по себе, это работающая программа, которая позволяет пользовательским программам выполняться и работать с файлами. А потоки позволяют коду выполняться параллельно. Мы будем отталкиваться от того, что файл, все же, это некие данные и метаданные, которые эти данные описывают (надо понимать, что этих данных может как таковых не быть).
\end{important}
