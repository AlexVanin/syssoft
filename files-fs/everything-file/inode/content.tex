Каждый файл в ОС UNIX описывается специальной структурой --- индексным дескриптором (inode). 

\begin{defi}{Индексный дескриптор}
	структура, описывающая файл, содержащая метаданные о файле (служебную информацию, необходимую для обработки данных).
\end{defi}

Приведем некоторы важные поля, содержащиеся в индексном дескрипторе:

\begin{itemize}
	\item \begin{defi}{mode}
		тип файла и права на него;
		\end{defi}

	\item \begin{defi}{nlink}
		количество жестких ссылок;
		\end{defi}

	\item \begin{defi}{uid}
		идентификатор пользователя -- владельца файла;
		\end{defi}

	\item \begin{defi}{gid}
		идентификатор группы -- владельца файла;
		\end{defi}

	\item \begin{defi}{size}
		размер файла в байтах;
		\end{defi}

	\item \begin{defi}{atime}
		Время последнего доступа;
		\end{defi}

	\item \begin{defi}{mtime}
		время последнего изменения содержимого;
		\end{defi}

	\item \begin{defi}{ctime}
		время последнего изменения метаданных;
		\end{defi}

	\item \begin{defi}{gen}
		генерируемый номер новой inode (инкриментируется при каждом запросе идентификатора для нового файла);
		\end{defi}

	\item \begin{defi}{addr[13]}
		адреса содержимого файла.
		\end{defi}
\end{itemize}

\begin{important}
	Индексный дескриптор не содержит имени файла и его содержимого.
\end{important}

На самом деле в ОС UNIX существуют три типа inode ---
\begin{itemize}
	\item inode --- хранится в оперативной памяти
	\item dinode --- хранится на диске
	\item vnode --- представлена в VFS (Виртуальной файловой системе). VFS нужна для обеспечения единообразного доступа к различным типам файловых систем.
\end{itemize}



