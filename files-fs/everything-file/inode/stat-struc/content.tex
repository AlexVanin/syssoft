Довольно часто нам требуется узнать  информацию о каком-то файле. Какого типа информацию? Например, что это за файл, его права доступа. Фактически, ту информацию, которая хранится в inode. И эту информацию мы можем получить с помощью системного вызова stat (2). Называется он от "get status file" (получить статус файла).

\begin{defi}{stat}
	структура для хранения метаданных файла, полученных системным вызовом stat (Сами метаданные хранятся в структуре индексного дескриптора).
\end{defi}

\textbf{Некоторые поля структуры stat}

\begin{center}
\begin{tabular}{l|l}

	\textbf{Поле}	&	\textbf{Назначение} \\
	\hline
	mode\_t st\_mode	&	Тип файла и права на него \\
	\hline
	ino\_t st\_ino	&	Номер индексного дескриптора \\
	\hline
	dev\_t st\_dev	&	Номер устройства, на котором хранится файл \\
	\hline
	NLink\_t st\_nlink	&	Количество жестких ссылок \\
	\hline
	uid\_t st\_uid	&	UID \\
	\hline
	gid\_t st\_gid	&	GID \\
	\hline
	off\_t st\_size	&	Размер файла в байтах \\
	\hline
	time\_t st\_atime		&	Время последнего доступа \\
	\hline
	time\_t st\_mtime		&	Время последней модификации содержимого \\
	\hline
	time\_t st\_ctime		&	Время последней модификации метаданных \\
	\hline
	long st\_blksize		&	Оптимальный размер для ввода-вывода \\
	\hline
	long st\_blocks		&	Число размещенных блоков хранения \\
	\hline

\end{tabular}
\end{center}

Для получения stat существует следующее семейство системных вызовов:

\begin{CCode}{stat (2)}
	#include <sys/types.h>
	#include <sys/stat.h>

	int stat(
		const char *path, 
		struct stat *st
	); \end{CCode}
Через имя файла (для обычного файла).
			
\begin{CCode}{lstat (2)}
	#include <sys/types.h>
	#include <sys/stat.h>
	
	int lstat(
		const char *path, 
		struct stat *st
	); \end{CCode}
Через имя файла (для символьной ссылки).
	
\begin{CCode}{fstat (2)}
	#include <sys/types.h>
	#include <sys/stat.h>
	
	int fstat(
		int filedes, 
		struct stat *st
	); \end{CCode}
Через файловый дескриптор

Эти системные вызовы возвращают ноль, в случае успеха, либо код ошибки.