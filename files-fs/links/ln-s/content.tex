Наряду с жесткими ссылками есть такое понятие, как символьные ссылки. Это абсолютно разные вещи. Жесткая ссылка --- это запись внутри каталога. Символьная ссылка --- это отдельный файл со своей собственной inode, с собственными блоками данных и типом symlink. Внутри символьной ссылки содержится некая строчка (абсолютно любая, какую запишем). Эту строчку при работе с символьной ссылкой ядро ОС будет считать путем, по которому нужно пройти к основному файлу.

Для создания символьной ссылки существует системный вызов symlink (2). Его аргументами являются путь к существующему файлу и путь к файлу-символьной ссылке, которую мы хотим создать. 

\begin{CCode}{symlink(2)}
	#include <unistd.h>

	int symlink(
		const char *name1, 
		const char *name2
	); \end{CCode}

Для чтения содержимого символьной ссылки существует readlink (2). Его аргументы --- путь к символьной ссылке, буфер для хранения прочитанного и размер буфера (количество байт, которые мы хотим прочитать). 

\begin{CCode}{readlink(2)}
	#include <unistd.h>

	ssize_t readlink( 
		const char *restrict path,	/* path to link */ 
		char *restrict buf,			/* buffer */ 
		size_t bufsiz				/* size of buffer */ 
	); \end{CCode}
