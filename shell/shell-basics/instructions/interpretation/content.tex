Коротко о том, как shell интерпретирует команду. После набора инструкции и нажатия клавиши ввода можно выделить следующие шаги:

\begin{enumerate}
	\item подстановка:
		\begin{enumerate}
			\item результатов команд, выполненных в подоболочке;
			\item позиционных параметров и переменных;
			\item вычисление арифметических выражений;
			\item путей (glob-джокеров);
			\item разбиение на слова (в соответствии с IFS).
		\end{enumerate}
	\item подстановка алиасов;
	\item поиск команды сначала по встроенным в shell, а в случае ненахождения --- по путям, указанным в переменной \$PATH;
	\item выполнение команд согласно расположению и свойствам разделителей команд.
\end{enumerate}

\begin{shCode}{Например}
	ag@helios:/home/ag$ ls -l $(echo mydir) && rmdir mydir \end{shCode}