Иногда есть необходимость обрезать часть значения переменной при её использовании. Для этого существуют символы подстроки \# и \%. Символ подстроки \# позволяет обрезать содержимое переменной слева, а символ \% - справа.

Запомнить можно так: \# расположена на клавиатуре слева, \% --- справа.

Рассмотрим, как это работает на примерах.

\begin{shCode}{Символ подстроки \#}
	ag@helios:/home/ag$ myvar="airport is good"
	ag@helios:/home/ag$ echo ${myvar#air}
	port is good \end{shCode}

Перед выводом значения переменной myvar shell увидит символ подстроки \# и исключит из вывода первые три символа --- “air“, если исходное значение переменной начинается с этих символов. 

\begin{shCode}{Символ подстроки \%}
	ag@helios:/home/ag$ myvar="airport"
	ag@helios:/home/ag$ echo ${myvar%port}
	air \end{shCode}

Перед выводом значения переменной myvar shell увидит символ подстроки \% и исключит из вывода последние четыре символа --- “port“, если исходное значение переменной заканчивается этими символами. 

\begin{shCode}{Не забывайте также о возможности использования glob-джокеров}
	ag@helios:/home/ag$ myvar="airport is good"
	ag@helios:/home/ag$ echo ${myvar%port*}
	air \end{shCode}
	
\begin{important}
Символы \% и \#, можно сказать, дадют только временный эффект. Значение самой переменной они не изменяют. Это можно сделать только использовав операцию присвоения.

\begin{shCode}{Например}
	ag@helios:/home/ag$ myvar="airport is good"
	ag@helios:/home/ag$ myvar=${myvar%port*} \end{shCode}
\end{important}
