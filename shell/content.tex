Многие из вас, уже успели поработать с shell и успели понять, что такое ввод команд и реакция системы на это. 

Сразу обозначим одну вещь --- если мы хотим программировать нужно использовать инструменты для программирования, а не shell. Shell, это командный интерпретатор, инструмент, который позволяет автоматизировать запуск команд, а не писать сложные алгоритмы. Чаще всего shell скрипт не должен быть интерактивным, а должен автоматизировать выполнение каких-то команд. При работе с командным интерпретатором нам не должно быть интересно постоянно инструктировать скрипт о том, что ему делать. 

В этом разделе мы рассмотрим базовые принципы взаимодействия с операционной системой UNIX посредством командного интерпретатора.