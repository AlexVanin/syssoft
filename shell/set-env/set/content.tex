\begin{defi}{set}
	встроенная в shell команда, она предназначена для установки или сброса ключей и позиционных параметров.
\end{defi}

\begin{shCode}{Например}
	ag@helios:/home/ag$ set word1 word2 word3 \end{shCode}
Установит значения 1-го, 2-го и 3-го позиционных параметров, соответственно.

Процесс экспортирования необратим. Единственный способ сделать эту переменную не экспортированной это удалить эту переменную т..е сделать операцию unset.

Приведем некоторые режимы set:
\begin{itemize}
	\item \textbf{set -v} -- выводит на терминал текст исходной команды, перед ее выполнением;
	\item \textbf{set -x} -- выводит на терминал текст интерпретированной команды, перед ее выполнением;
	\item \textbf{set -vi/emacs} -- устанавливает в shell режим управления как в соответствующих редакторах;
	\item \textbf{set -o ignore EOF} -- игнорировать конец файла;
	\item \textbf{set -o no exec} -- запретить выполнение команд в текущем shell;
	\item \textbf{set -o no clobber} -- аналог >|.
\end{itemize}

\begin{important}
	Символ - перед опцией означет включение режима, символ + -- выключение
\end{important}