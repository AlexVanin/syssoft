Говоря о позиционных параметрах, нельзя забывать две переменных окружения ---  “@” и “*”. Это такие переменные, которые  позволяют обратиться ко всем позиционным параметрам, начиная с первого.

При написании без двойных кавычек (echo \$* и echo \$@) эти переменные выдают одинаковый результат - подставляют вместо себя позиционные параметры разделенные пробелом. Разница станет значительной лишь при использовании двойных кавычек:

echo “\$*” 	выведет одну большую строку в кавычках, разделенную первым символом из IFS.
 
echo “\$@” 	    выдаст набор разделенных пробелом позиционных параметров (по сути массив,с которым мы сможем работать). Этот момент всегда нужно помнить. 

Рассмотрим пример с установкой значения IF в :

\begin{shCode}{Файл myscript}
#!/bin/sh
IFS=:
echo "$*"
echo "$@" \end{shCode}

\begin{shCode}{shell}
	ag@helios:/home/ag$ ./myscript test 67 value
	test:67:value
	test 67 value \end{shCode}

На самом деле, специальных переменных довольно много, рассмотрим некоторые из них:

\begin{myenv}{\$-}{содержит ключи, передаваемые shell (фактически опции задаваемые с помощью set).}
\end{myenv}

\begin{myenv}{\$\_}{содержит последний аргумент предыдущей выполненной команды.}
\end{myenv}

\begin{myenv}{\$\#}{содержит число позиционных параметров.}
\end{myenv}

\begin{myenv}{\$?}{содержит код возврата предыдущей команды.}
\end{myenv}

\begin{myenv}{\$!}{содержит идентификатор процесса, который был последним свернут в бэкграунд группу. Если вы запустили два процесса, то в переменной \$! будет PID последнего процесса, который был запущен.}
\end{myenv}
