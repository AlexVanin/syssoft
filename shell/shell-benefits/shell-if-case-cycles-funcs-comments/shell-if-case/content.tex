Для ветвления программ в shell существуют условные конструкции IF и CASE.

\begin{shCode}{IF...THEN...ELSE}
	IF ( <condition> ) 
		THEN <actions>
	ELIF ( <alternative condition> ) 
		THEN <actions>
	ELSE 
		<actions>
	FI
\end{shCode}

В качестве условия выполнения --- <condition> --- принимает набор команд (Например, через  \&\&). Действие после THEN будет выполнено только в случае, если код возврата последней команды --- успех.

\begin{important}
	Помним, что при написании IF и THEN в одной строке они должны разделяться “;”, так как являются разными командами
\end{important}

\begin{shCode}{То есть:}
	IF ( <condition>) ; THEN <Actions>
	ELIF ( <alternative condition> ) ; THEN <actions>
	ELSE <actions>
	FI
\end{shCode}


\begin{shCode}{CASE}
	CASE (<string>) IN
     pattern-1)      
		<actions>
          	;;
     pattern-2)      
     	<actions>
          	;;
     pattern | pattern)
     	<actions>
          	;;
	ESAC
\end{shCode}

Принимает строку и в случае соответствие ее одному из установленных шаблонов совершает необходимые действия. По сути, аналог switch.

\begin{important}
	Не забываем о возможностях использования glob-джокерах в CASE. Так * будет обозначать не вошедшие в шаблоны случаи.
\end{important}

Заметка: завершающие слова в IF и CASE - просто написание их же в обратном порядке (IF - FI, CASE - ESAC)

