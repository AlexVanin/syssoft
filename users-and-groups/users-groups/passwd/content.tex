В ОС UNIX пользователь представляет собой некоторую сущность, которая обладает некими атрибутами. Атрибуты каждого пользователя можно увидеть в файле /etc/passwd. Этот файл хранит информацию о пользователях системы.

Приведем список атрибутов пользователя.

\textbf{name : password : UID : GID : GECOS: home : shell}

\begin{itemize}

	\item \begin{defi}{name} 
			логин;
			\end{defi}
	
	\item \begin{defi}{x} 
			пароль (сейчас перемещен например в /etc/master.passwd (FreeBSD) или /etc/shadow);
			\end{defi}
			
	\item \begin{defi}{UID}
			идентификатор пользователя (обычно от 0 до 65535);
			\end{defi}
			
	\item \begin{defi}{GID}
			идентификатор главной группы;
			\end{defi}
			
	\item \begin{defi}{GECOS}
			вспомогательная информация о пользователе (например, имя, номер телефона, адрес).
			
			Для изменения GECOS используется утилита chnf;
			\end{defi}
			
	\item \begin{defi}{home}
			домашний каталог пользователя;
			\end{defi}
			
	\item \begin{defi}{shell} 
			командный интерпретатор, запускаемый по умолчанию.
			\end{defi}
			
\end{itemize}
