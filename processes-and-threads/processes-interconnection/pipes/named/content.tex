Именованные каналы во многом работают так же, как и неименованные каналы, но все же имеют несколько заметных отличий.

\begin{itemize}
	\item именованные каналы существуют в виде специального файла устройства в файловой системе;
	\item процессы различного происхождения могут разделять данные через такой канал;
	\item именованный канал остается в файловой системе для дальнейшего использования и после того, как весь ввод/вывод сделан.
\end{itemize}


Существует два способа создания именованного канала:

Создать обычный файл, директорию или файл специального назначения с помощью системного вызова mknode (2), указав 0 в dev\_t.

\begin{CCode}{mknode (2)}
	int mknod(
		const char *path, 
		mode_t mode, 
		dev_t dev
	); \end{CCode}

или воспользоваться функцией mkfifo (3)

\begin{CCode}{mkfifo (3)}
	int mkfifo(
		const char *pathname, 
		mode_t mode
	); \end{CCode}
