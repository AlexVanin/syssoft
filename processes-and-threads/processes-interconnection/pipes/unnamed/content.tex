Для создания неименованного калала существет системные вызовы pipe (2) и pipe2 (2). Оба они принимают на вход массив из двух элементов. После успешного выполнения вызова массив содержит два файловых дескриптора: для чтения информации из канала и для записи в него соответственно. 

\begin{CCode}{pipe(2)}
	#include <unistd.h>
	
	int pipe(
		filedes[2]
	); \end{CCode}

Системный вызов pipe2 (2) принимает также некоторые флаги, влияющие на поведение канала.

Неименованные каналы можно использовать для родственных процессов. Когда процесс порождает другой процесс, дескрипторы родителя наследуются дочерним процессом, и, таким образом, осуществляется связь между двумя процессами. Один из них использует канал только для чтения, а другой только --- для записи.