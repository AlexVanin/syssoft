На все сигналы, кроме SIG\_STOP и SIG\_KILL, вы можете самостоятельно написать свой обработчик сигналов --- собственную функцию, которая будет реагировать на пришедший сигнал так, как вы в ней описали.

Свой обработчик сигналов можно сделать разными способами.

Библиотека libc предлагает механизм, который называется signal (3). Эта функция принимает в качестве аргементов номер сигнала и новую диспозицию сигнала. В качестве диспозиции можно подать функцию, которую вы хотите назначить обработчиком сигнала. Функция возвращает предыдущую диспозицию.

\begin{CCode}{signal (3)}
	#include <signal.h>

    void (*signal(int sig, void (*disp)(int)))(int); \end{CCode}
~\\[0.5cm]

\textbf{Какие недостатки есть у signal (3)?}

Во многих реализация libC диспозиция сигнала устанавливается на действие по умолчанию каждый раз при получении сигнала.

Кроме того, у процесса в контексте есть такое понятие, как маска принимаемых сигналов. Фактически, это битовая последовательность, которая маскирует сигналы от принятия при их передаче. Если в этой битовой последовательности первый бит установлен в единицу, то, при попытке доставить сигнал процессу, операционная система увидит, что этот сигнал замаскирован, и поэтому его доставлять не нужно --- он будет просто отброшен.

Есть еще один более гибкий традиционно используемый способ обработки сигналов --- это системный вызов sigaction (2), который принимает числовой номер сигнала и два указателя на структуру sig\_action.

Первый указатель нужен для того, чтобы указать ОС, что мы хотим делать в случае прихода этого сигнала т.е. Во второй указатель помещается реакция на сигнал после выполнения системного вызова sigaction (2).

\begin{CCode}{sigaction (2)}
	#include <signal.h>

     int sigaction(int sig, const struct sigaction *restrict act,
         struct sigaction *restrict oact);  \end{CCode}

Структура sigaction включает в себя следующие поля:

\begin{itemize}
	\item void (*sa\_handler)(int) -- указатель на функцию обработчик сигнала;
	\item void (*sa\_sigaction)(int, siginfo\_t *, void *) -- указатель на функцию обработчик сигнала, если установлен флаг SA\_SIGINFO;
	\item sa\_flags -- флаги;
	\item sa\_mask -- битовая маска, которая маскирует прием/доставку сигналов процессу.
\end{itemize}
