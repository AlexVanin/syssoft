\begin{defi}{Семафор}
	это объект синхронизации, имеющий множество состояний: 0(заблокирован), 1, 2, …, N
\end{defi}

Семафор представляет из себя счетчики, ограничивающий количество возможных одновременных пользователей ресурса. Значение семафора отражает количество свободных мест.

Семафоры находятся в адресном пространстве ядра, поскольку доступ к ним должны иметь все процессы.

\textbf{Примерный алгоритм работы с семафором}

Изначально семафору устанавливается целое положительное значение N --- максимальное значение одновременных пользователей какого-либо ресурса. 

Когда новый пользователь хочет захватить (начать работать с ресурсом), он уменьшает значение семафора на 1. Таким образом уже N-1 пользователей могут начать использовать ресурс в данный момент. После окончания работы с ресурсом, пользователь должен увеличить значение N на 1, тем самым освободив место для еще одного пользователя. 

Если значение N становится равным 0, это значит, что уже никто не может получить доступ к желаемому ресурсу.

Существуют POSIX и System V семафоры. Семафоры System V являются не отдельными счетчиками, а представляют из себя группу счетчиков, объединенных каким-либо признаком.

\begin{CCode}{Для работы с семафорами System V используются такие системные вызовы как:}
	#include <sys/sem.h>

	int semctl(int semid, int semnum, int cmd, ...);

	int semget(key_t key, int nsems, int semflg);

	int semids(int *buf, uint_t nids, uint_t *pnids);

	int semop(int semid, struct sembuf *sops, size_t nsops);

	int semtimedop(int semid, struct sembuf *sops, unsigned nsops,
			struct timespec *timeout); \end{CCode}

\begin{CCode}{Для работы с семафорами POSIX используются такие функции как:}
	#include <semaphore.h>

	int sem_wait(sem_t *sem);

	int sem_init(sem_t *sem, int pshared, unsigned int value);

	int sem_post(sem_t *sem);

	int sem_getvalue(sem_t *sem, int *sval);

	int sem_destroy(sem_t *sem); \end{CCode}

Вам предлагается ознакомиться с ними самостоятельно.