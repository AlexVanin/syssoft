Как мы помним, все процессы кроме init порождаются системными вызовами fork(2) или vfork(2). 

Системный вызов fork (2) создает дочерний процесс - точную копию родительского процесса. При успешном выполнении процессу родителю возвращается PID потомка, а потомку - 0.

\begin{CCode}{fork (2)}
	#include <sys/types.h>
	#include <unistd.h>
	
	pid_t fork(void); \end{CCode}

\begin{CCode}{Пример}
	int main(void) {
		pid_t pid;

		pid = fork(); /* change to vfork */
	
	    if (pid == 0) {
	        /* Now we're in child process */
		}   
		else {
			/* Now we're in parent process */
		}
		return EXIT_SUCCESS; 
	} \end{CCode}


Системный вызов vfork (2) определяется как fork (2) со следующим ограничением: поведение функции не определено, если созданный с её помощью процесс совершит хотя бы одно из следующих действий:

\begin{itemize}
	\item Произведет возврат из функции, в которой был вызван vfork (2);
	\item Вызовет любую функцию кроме \_exit() или exec* (2);
	\item Изменит любые данные кроме переменной, в которой хранится возвращаемое функцией vfork (2) значение.
\end{itemize}

\begin{CCode}{vfork (2)}
	#include <sys/types.h>
	#include <unistd.h>

	pid_t vfork(void); \end{CCode}

~\\~\\
\textbf{Разница между fork (2) и vfork (2).}

В отличие от fork (2), vfork (2) не создает копию родительского процесса, а создает разделяемое с родительским процессом адресное пространство до тех пор, пока не будет вызвана функция \_exit или одна из функций exec.
Родительский процесс на это время останавливает свое выполнение. Отсюда следуют и все ограничения на использование – дочерний процесс не может изменять никакие глобальные переменные или даже общие.

Другими словами, после вызова vfork() оба процесса будут видеть одни и те же данные и переменные.

Единственный вариант использования vfork --- когда вы предполагаете, что в скором будущем вы выполните функцию exec, которая заместит ваше адресное пространство новым исполняемым файлом. Соответственно, начиная с этого момента, будет разблокировано выполнение процесса родителя.

\begin{CCode}{Пример}
	int main(void) {
		pid_t pid;
		int status, common_var = 0;
	
		pid = fork(); /* change to vfork */
	
	    if (pid == 0) {
	        /* Child  */
			common_var = 1;
			printf("Child common_var value: %i \n",  common_var);
			
			exit(EXIT_SUCCESS);
		}   
		waitpid(pid, &status, 0);
		printf("Parent common_var value: %i \n",  common_var);
		
		if (common_var) {
			puts("vfork(): common variable has been changed");
		} else {
			puts("fork(): common variable hasn't been changed");
		}   
		return EXIT_SUCCESS; 
	} \end{CCode}

\begin{CCode}{Вывод для fork (2)}
	Child common_var value: 1 
	Parent common_var value: 0 
	fork(): common variable hasn't been changed \end{CCode}

\begin{CCode}{Вывод для vfork (2)}
	Child common_var value: 1 
	Parent common_var value: 1 
	vfork(): common variable has been changed \end{CCode}
