Приведем заголовочные файлы, наиболее часто используемые в системном программировании в операционной системе UNIX.

	\begin{center}
		\begin{tabular}{l|l}	
			\textbf{Имя} & \textbf{Содержимое} \\
			\hline
			unistd.h 	& объявления UNIX \\
			\hline
			stdio.h  	& стандартный ввод/вывод \\
			\hline
			fcntl.h  	& операции с файлами (например: open) \\
			\hline
			sys/types.h	& системные типы \\
			\hline
			sys/stat.h	& системные статусы \\
			\hline
			errno.h	& errno и и директивы с определением ошибок \\
		\end{tabular}
		\end{center}
		
Более подробно узнать об их назначении можно узнать использовав команду \textbf{man}.

	\begin{shCode}{Например}
		ag@helios:/home/ag$ man stdlib 
		ag@helios:/home/ag$ man stdlib.h \end{shCode}
