В этой главе мы рассмотрим некоторые особенности системного программирования на языке Си, структуру С-программы и способы уточнения причин ошибок, возникающих в ходе ее работы.

В качестве основного примера возьмем следующую программу:

\begin{CCode}{Листинг main.c}
			#include <stdlib.h>
			#include <fcntl.h>
			#include <errno.h>			
		
			int main(int argc, char *argv[], char *envp[]) { 
				
				if (open(argv[1], O_RDONLY) < 0) {
					perror("main");			
					return EXIT_FAILURE;
				}	 
				
 				close(argv[1]);
 				
 				return EXIT_SUCCESS;
			} 
\end{CCode}

Ее назначение --- определить существует ли доступный для чтения файл с именем, поданным первым аргументом на вход программе.
